\documentclass[12pt]{article}

\usepackage{sbc-template}

\usepackage{graphicx,url}
\usepackage{amssymb}

%\usepackage[brazil]{babel}   
%\usepackage[latin1]{inputenc}  
\usepackage[utf8]{inputenc}  
% UTF-8 encoding is recommended by ShareLaTex

     
\sloppy

\title{Implementation and Analysis of Collaborative Filtering Algorithm on MovieLens Latest Small Dataset}

\author{Xinqi Zhu 5150297, Ryan Mckay z5060961}

\address{School of Computer Science and Engineering -- University of New South Wales}


\begin{document} 

\maketitle

\begin{abstract}
We present our study of recommender system's and user-item rating systems, i.e. user-movie rating system. We have focused on the most prolific algorithm in recommender systems, collaborative filtering, more specifically Latent Factor and K-Nearest-Neighbour (KNN) techniques for rating prediction and we establish a simple classification rule as a means for considering so called top N recommendations. In this report, we outline the implementation, experimentation and evaluation of our collaborative filtering solution and explore various methods to measure prediction accuracy based on experiments. 
\end{abstract}


\section{Introduction}

Recommender system's have become very popular in recent times, implemented by many Internet giants to guide customers toward products they are likely to click on, watch, purchase or consume in general. One of the most successful recommendation algorithms is collaborative filtering.

%In this project, our goal is to implement a version of collaborative filtering algorithm on a MovieLens user-movie rating dataset and do sufficient experiments to make the system work properly and give a reasonable prediction accuracy.
%
%Also we apply multiple evaluation methods to our system and see how the value of parameters change the performance of the system.

%\section{Related Work}


\section{Data Preparation}

Firstly we performed data cleaning, which included handling null values and transforming the raw data into a matrix form (movieId x userId) with new sequential indexing so that we could perform matrix factorisation and compare user and movie vectors easily. We also recorded a rated matrix which held boolean values indicating if a particular user had rated a movie.

Next we created two methods for partitioning our data random sample and temporal split. Random Sample ignores the timestamp value and takes the supplied proportion for the holdout test set as a random sample using scikit-learns ShuffleSplit. Temporal split sorts the data by timestamp and partitions the data such that the supplied proportion for the holdout set is the most recent subset of user ratings. 

The temporal split is a means of naively simulating an online recommender system as often the system will be given a new user and rating to predict at a moment in time.

\section{Implementation}

\subsection{Collaborative Filtering Algorithm}
There are two primary areas of collaborative filtering: neighborhood methods and latent factor models \cite{MF}. Our implementation includes: 
\begin{itemize} 
	\item Latent factor model (Matrix Factorisation)
	\item KNN model
	\item K-Fold cross validation
	\item Precision-Recall Curve 
	\item Performance evaluation metrics and supporting visualizations
\end{itemize}

\subsubsection{Latent Factor Model}
This algorithm will model the user-movie rating matrix as inner product of a user matrix and a movie matrix. Each movie is represented as a vector $X_i \in \mathbb{R}^f$ and each user is associated with $\theta_j \in \mathbb{R}^f$. The approximate rating by user j on movie i is:
$$\hat{r}_{ij} = X_i^T \cdot \theta_j$$

This implementation requires the determination of the number of features $f$, which is used to represent the features of a movie and the preference of a user on these features. We discuss selection of numbers of features in Experimental Evaluation (section 4).

In order to get the proper $X$ and $\theta$ matrix, we use the definition of cost function from \cite{MF}:
$$J = \frac{1}{2}\sum_{(i,j):t(i,j)=1}(X_i^T \cdot \theta_j - r_{ij})^2+\frac{\lambda}{2}\sum_{i = 1}^{n_m}\sum_{k = 1}^{f}X_{ik}^{2}+\frac{\lambda}{2}\sum_{j = 1}^{n_u}\sum_{k = 1}^{f}\theta_{jk}^{2}$$
where $t(i,j)$ means movie $i$ has been rated by user $j$. $\lambda$ is the regularization coefficient and should be determined by experiment.

The gradient of $X$ and $\theta$:
$$\frac{\partial J}{\partial X_{ik}} = \sum_{j:t(i,j)=1}(X_i^T \cdot \theta_j - r_{ij})\theta_{jk}+\lambda X_{ik}$$
$$\frac{\partial J}{\partial \theta_{jk}} = \sum_{i:t(i,j)=1}(X_i^T \cdot \theta_j - r_{ij})X_{ik}+\lambda \theta_{jk}$$

Then we use the predefined optimization method of Conjugate Gradient from scipy library to train our model. In order to do so, we compress the $X$ and $\theta$ matrix into a vector to fit the minimization function.

To turn our matrix factorisation into prediction we take the inner product of $X$ and $\theta$ matrix resulting in $\hat{r}$ matrix and round each value to the nearest rating level (0.5, 1.0, 1.5 ...). $\hat{r_{ij}}$ is our prediction of user $j$ rating of movie $i$ and we compare this prediction to $r_{ij}$ to evaluate our models predictive performance. We discuss this further in section 5.

\subsubsection{K-Nearest-Neighbor Model}

We explore KNN as another method of prediction based on our Latent Factor model. Using $\theta$ matrix from our Latent Factor model we compute similarity of each user in our test set to the subset of users that rated the movie that we wish to predict for the test set user.  
\\ \\
We compute "similarity" with two measures, Pearson Correlation and Cosine similarity as defined below.
\\
Pearson Correlation $r_{p}$ of users x and y with feature vectors $r_{x}$ and $r_{y}$ respectively with $ith$ feature from $\theta$ matrix.  
$$r_{p} = \dfrac{ \sum_{i\in} (r_{x,i} - \bar{r_{x}})(r_{y,i} - \bar{r_{y}})}{\sqrt{\sum_{i} (r_{x,i} - \bar{r_{x}})^{2} \sum_{i} (r_{y,i} - \bar{r_{y}})^{2}}}$$ 
\\ \\
Cosine Similarity is the cosine of the angle between user feature vectors x and y from $\theta$ matrix.
$$r_{c} = \cos(\textbf{x},\textbf{y}) = \frac{\sum_{i} r_{x,i} r_{y,i} }{\sqrt{\sum_{i} r_{x,i}^2} \sqrt{\sum_{i} r_{y,i}^2}} $$
\\
From the subset of users that rated the test set movie, we select the k most similar users to the test set user by the latent features from $\theta$ matrix and the similarity measures described above. Our prediction for the test set users rating of a particular movie is the mean of the ratings given by its K nearest users. When no users had previously rated the movie from the test set, we took a simple average of all of the user in questions ratings. 

\subsection{K-fold Cross Validation}
We use k-fold cross validation (CV) to determine the proper number of features $f$, the regularization coefficient $\lambda$ and do any other debugging.

By using the KFold function of sklearn's model\_selection library, we randomly sampled (time independent) and split the validation set into k folds and use each of them as a holdout testing set while the Latent Factor Model trains on the remaining folds. In our implementation, we take $k=5$ and use the mean of the Mean Squared Error (MSE) and Mean Absolute Error (MAE) from all iterations of our Cross Validation.

\section{Experimental Evaluation}
The evaluation metrics we considered are Mean Squared Error (MSE) and Mean Absolute Error (MAE) for rating predictions and Precision and Recall for our classification (Top N Recommender). We chose to use MAE as well as the common MSE as it is more sensitive to outliers and since we have only simple methods for dealing with the cold start problem we believed the comparison of results may be relevant.

\subsection{Latent Factor Model}
We randomly split the raw data of user-movie rating into training set (80\%) and test set (20\%), with the training set as a validation set we used k-fold cross validation with $k=5$. From the cross validation we determined the number of features $f$, regularization coefficient $\lambda$, produced a Learning Curve Figure \ref{fig:exampleFig1} and Receiver Operating Characteristic (ROC) curve Figure \ref{fig:exampleFig1}. 

Figure \ref{fig:exampleFig1} and Figure \ref{fig:exampleFig1} are the results from the experiment to find the optimal $f$ and $\lambda$, which resulted in a selection of $f=105$ and $\lambda = 1.3$. We can see when $\lambda$ is larger than 1.3 the MSE on CV set is increasing and likely to be underfitting (high bias). On the contrary, low $\lambda$ results in the system overfitting (high variance). We set $f=105$ at the elbow point because larger $f$ takes more time to train but has no obvious improvement on result, and lower $f$ suffers with higher error.

Figure \ref{fig:exampleFig1} shows the learning curve under three conditions: underfitting ($\lambda = 5$), normal ($\lambda = 1.3$), and overfitting ($\lambda = 0.2$).

Figure \ref{fig:exampleFig1} shows the ROC curve of our recommender system. The confusion matrix is defined based on the assumption that the truth rating beyond a threshold should be regarded as true positive and below as true negative\cite{CF_Recsys_Survey}. For prediction, we maintain a top rating list, i.e. top 100 rating movies to be predicted positive and rest to be predicted negative. We test three thresholds, 3, 4 and 4.5 stars. We draw the curve by changing the size of the top N list.

\subsection{KNN Model}
For KNN we attempted to understand if it could improve on the latent factor model by using its derived features as a measure of user similarity. We attempted a simple KNN experiment on the raw 671 user x 9066 movie matrix, but abandoned it after an extreme processing time. We would have used this result as a base line, both for our Latent Factor Model and further KNN experiments using the Latent features.

The experiments conducted with our KNN implementation were to understand the effect of the value of k and the sampling method on MSE and MAE.

\subsection{Results}

For latent factor model, our system achieved a MSE of 0.9 and MAE of 0.5. The average accuracy of correct prediction on rated movies is 70\%. This is an amazing ratio because a majority of the user's interaction with movies has been correctly predicted. The ROC curve shows that the prediction is much better than random guess.

For KNN model using a random sample to partition the data we observed a MSE of ~1 (Figure \ref{fig:knn1}) and MAE of ~0.765 (Figure \ref{fig:knn2}) and using a temporal split of the data we observed a best MSE of ~1.16 (Figure \ref{fig:knn3}) and MAE of 0.84 (Figure \ref{fig:knn4}). We observed similar results for calculating similarity with cosine and pearson correlation measures. In all cases cosine measure was superior, particularly for lower values of k.

Considering the effect of k on our metrics from Figures \ref{knn1} - \ref{knn4} we can conclude that they converge quite quickly and that k > 10 for the random split and k > 5 for the temporal split yielded no specific improvement. We should consider that with a larger dataset we will find a larger number and more specifically similar users and hence k may not converge so quickly. 

In regards to the temporal split convergence we must consider the temporal distribution of the data as movies could be rated at similar, as may users rate many movies at similar times. We hypothesise that this could enhance the cold start problem producing a higher error, but also producing a quicker convergence of k as a user in the test partition may have sparse entries in the training partition and would result in inaccurate similarity scores.    
 
\section{Future Work}

\section{Conclusion}

%Bibliographic references must be unambiguous and uniform.  We recommend giving
%the author names references in brackets, e.g. \cite{knuth:84},
%\cite{boulic:91}, and \cite{smith:99}.
%
%The references must be listed using 12 point font size, with 6 points of space
%before each reference. The first line of each reference should not be
%indented, while the subsequent should be indented by 0.5 cm.

\bibliographystyle{unsrt}
\bibliography{sbc-template}

\end{document}
